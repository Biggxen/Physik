% zeichnet einen Würfel mit gegebener Kantenlänge
\newcommand{\ZeichneWuerfel}[1]{% kantenlaenge
    \begin{tikzpicture}[]
        \def\a{#1}
        
        \coordinate (A) at (0,0,0); 
        \coordinate (B) at (\a,0,0) ;
        \coordinate (C) at (\a,\a,0); 
        \coordinate (D) at (0,\a,0); 
        \coordinate (E) at (0,0,-\a); 
        \coordinate (F) at (\a,0,-\a); 
        \coordinate (G) at (\a,\a,-\a); 
        \coordinate (H) at (0,\a,-\a);
        % Kanten zeichnen
        \draw[] (A) -- (B) -- (C) -- (D) -- (A);
        \draw[] (B) -- (F) -- (G) -- (C);
        \draw[] (G) -- (H) -- (D);
        \draw[densely dashed] (A) -- (E) -- (F);
        \draw[densely dashed] (E) -- (H);
    \end{tikzpicture} 
}

% zeichnet eine Pyramide mit quadratischer Grundfläche
\newcommand{\ZeichneQuadratischePyramide}[2]{% kantenlaenge, hoehe
    \begin{tikzpicture}[]
        \def\a{#1}
        \def\h{#2}
        
        \coordinate (A) at (0,0,0); 
        \coordinate (B) at (\a,0,0) ;
        \coordinate (C) at (\a,0,-\a); 
        \coordinate (D) at (0,0,-\a); 
        \coordinate (S) at (\a,\h,\a); 
        % Kanten zeichnen
        \draw[] (A) -- (B) -- (C);
        \draw[] (A) -- (S);
        \draw[] (B) -- (S);
        \draw[] (C) -- (S);
        \draw[densely dashed] (C) -- (D) -- (A);
        \draw[densely dashed] (D) -- (S);
    \end{tikzpicture} 
}

% Amperemeter ohne Pfeil
\def\pgf@circ@myamperemeter@path#1{\pgf@circ@bipole@path{myamperemeter}{#1}}
\tikzset{myamperemeter/.style = {\circuitikzbasekey, /tikz/to
                               path=\pgf@circ@myamperemeter@path}}
\pgfcircdeclarebipole{}{\ctikzvalof{bipoles/amperemeter/height}}{myamperemeter}
    {\ctikzvalof{bipoles/amperemeter/height}}{\ctikzvalof{bipoles/amperemeter/width}}
{
  \def\pgf@circ@temp{right}
  \ifx\tikz@res@label@pos\pgf@circ@temp
    \pgf@circ@res@step=-1.2\pgf@circ@res@up
  \else
    \def\pgf@circ@temp{below}
    \ifx\tikz@res@label@pos\pgf@circ@temp
      \pgf@circ@res@step=-1.2\pgf@circ@res@up
    \else
      \pgf@circ@res@step=1.2\pgf@circ@res@up
    \fi
  \fi

  \pgfpathmoveto{\pgfpoint{\pgf@circ@res@left}{\pgf@circ@res@zero}}     
  \pgfpointorigin   \pgf@circ@res@other =  \pgf@x
  \advance \pgf@circ@res@other by -\pgf@circ@res@up
  \pgfpathlineto{\pgfpoint{\pgf@circ@res@other}{\pgf@circ@res@zero}}
  \pgfusepath{draw}

  \pgfsetlinewidth{\pgfkeysvalueof{/tikz/circuitikz/bipoles/thickness}
                   \pgfstartlinewidth}

  \pgfscope
    \pgfpathcircle{\pgfpointorigin}{\pgf@circ@res@up}
    \pgfusepath{draw}       
  \endpgfscope  

  \pgfsetlinewidth{\pgfstartlinewidth}
  \pgftransformrotate{90} % rotate the label
  %\pgfpathmoveto{\pgfpoint{-\pgf@circ@res@other}{.8\pgf@circ@res@up}}
  %\pgfpathlineto{\pgfpoint{\pgf@circ@res@other}{.8\pgf@circ@res@down}}
  %\pgfusepath{draw}
  \pgfnode{circle}{center}{\textbf{A}}{}{}
  \pgfscope
    % \pgftransformshift{\pgfpoint{-\pgf@circ@res@other}{.8\pgf@circ@res@up}}
    % \pgftransformrotate{45}
    % \pgfnode{currarrow}{center}{}{}{\pgfusepath{stroke}}
  \endpgfscope

  \pgfpathmoveto{\pgfpoint{-\pgf@circ@res@other}{\pgf@circ@res@zero}}
  \pgfpathlineto{\pgfpoint{\pgf@circ@res@right}{\pgf@circ@res@zero}}
  \pgfusepath{draw}

  \pgfusepath{stroke}   
}

% Kreuze (z.B. für Magnetfeld)
\tikzset{cross/.style={cross out, draw=black, minimum size=2*(#1-\pgflinewidth), inner sep=0pt, outer sep=0pt},
    %default radius will be 1pt. 
    cross/.default={1pt}}
